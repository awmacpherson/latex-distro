%
%%  THEOREM ENVIRONMENTS
%
%%  v1.0.0
%

\usepackage{amsthm} % Provide proof environment, QED stack.
\swapnumbers        % Theorem number precedes heading.

% The following resets \end of theorem and proof environments to LaTeX
% defaults, which suppress indentation on immediately following line.

\makeatletter
\patchcmd{\@endtheorem}{\@endpefalse }{}{}{} 
\patchcmd{\endproof}{\@endpefalse}{}{}{}
\makeatother


% FUNCTION \DeclareTheorem[style = last_used]{name}{heading}
%
% Provides starred unnumbered version

\newcommand{\DeclareTheorem}[3][]{
  \ifstrempty{#1}{
    \newtheorem{#2}[main]{#3}
    \newtheorem*{#2*}{#3}
  }{
    \theoremstyle{#1}
    \newtheorem{#2}[main]{#3}
    \newtheorem*{#2*}{#3}
  }
}


% FUNCTION \InitThm{section_counter}
%
% Called in the preamble of documents to initialise
% theorem environments numbered within section_counter.
% Empty parameter yields absolute numbering.

\newcommand{\InitThm}[1]{
  \newtheorem{main}{}[#1]
  %
  \DeclareTheorem[plain]{theorem}{Theorem}
  \DeclareTheorem{thm}{Theorem}
  \DeclareTheorem{proposition}{Proposition}
  \DeclareTheorem{prop}{Proposition}
  \DeclareTheorem{lemma}{Lemma}
  \DeclareTheorem{corollary}{Corollary}
  \DeclareTheorem{cor}{Corollary}
  \DeclareTheorem{conjecture}{Conjecture}
  \DeclareTheorem{conj}{Conjecture}
  \DeclareTheorem{hypothesis}{Hypothesis}
  %
  \DeclareTheorem[definition]{para}{}
  \DeclareTheorem{definition}{Definition}
  \DeclareTheorem{defn}{Definition}
  %
  \DeclareTheorem[remark]{remark}{Remark}
  \DeclareTheorem{rmk}{Remark}
  \DeclareTheorem{example}{Example}
  \DeclareTheorem{eg}{Example}
}